\section{Simulation Analysis}
\label{sec:simulation}


\begin{table}[h]
  \centering
  \begin{tabular}{|l|r|}
    \hline    
    {\bf Name} & {\bf Value [A or V]} \\ \hline
    \input{../sim/table_tab}
  \end{tabular}
  \caption{Operating point. A variable preceded by @ is of type {\em current}
    and expressed in Ampere; other variables are of type {\it voltage} and expressed in
    Volt.}
  \label{tab:tabela3}
\end{table}



Table~\ref{tab:tabela3} shows the simulated operating point results for the circuit
under analysis.
Observing the theoretical and experimental results, we verify that the error between both results is 0 because 
they present the same results and that comes from the fact that the simulator uses the same models (mesh method and node method) to the calculations of the various quantities and
due to all the components have a linear behaviour. 



