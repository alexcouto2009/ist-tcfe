\newpage
\section{Introduction}
\label{sec:introduction}



The objetive of this laboratory assignment is to project a bandpass filter (BPF) using an OP-AMP, resistors and capacitors.
A bandpass filter is a basic circuit that only allows a certain range of frequencies to pass while blocking higher and lower frequencies.
In practice, the non-blocked frequencies will be amplified while the others won't.
As in the previous laboratory the main challenge was to seek for the best balance between quality and cost, because the result will be the quocient of those quantities.

In this presentation we will start showing (in section \ref{sec:analysis}) an illustration of the chosen configuration to the bandpass filter (Image \ref{fig:lab4}) and
then we present the circuit description. 
After that, we introduce our results such as the important comments, this is presented in the section \ref{sec:circuit}. This section (section \ref{sec:circuit}) is divided 
in three important subsections, 
in the subsection \ref{sub:t1} is where are presented the theoretical analysis of the circuit and then, in the subsection \ref{sub:s1}, is presented the simulation analysis
 of the circuit. In the other subsection (subsection \ref{sub:comp}) is where the comparison between the results provided by the theoretical analysis, using octave, and 
the simulations results, using ngspice, is done as well as the discussion of the results.

As in the last assignment we tested more than one configuration so it is possible to conclude some aspects from experience and that is presented in the section 
\ref{sec:aspects}.

To conclude is presented a final discussion about this laboratory assignment that can be seen in the section \ref{sec:conclusion}.





% state the learning objective 
%The objective of this laboratory assignment is to project an audio amplifier circuit and to do it we used resistors, capacitors and transistors.
%An audio amplifier is a basic circuit configuration that amplifies the signal received and then send it to a speaker.
%As in the previous laboratory the main challenge was to seek for the best balance between quality and cost, because the result will be the quocient of those quantities.

%In this presentation we will start showing an illustration of the chosen configuration to the audio amplifier circuit (Image \ref{fig:lab4}) and then we present the circuit 
%description and the explanation of why it was chosen. 
%After that, we introduce our results such as the important comments, this is presented in the section \ref{sec:circuit}. This section (section \ref{sec:circuit}) is divided 
%in three important subsections, 
%in the subsection \ref{sub:t1} is where are presented the theoretical analysis of the circuit and then, in the subsection \ref{sub:s1}, is presented the simulation analysis
% of the circuit. In the other subsection (subsection \ref{sub:comp}) is where the comparison between the results provided by the theoretical analysis, using octave, and 
%the simulations results, using ngspice, is done as well as the discussion of the results.


%As in the last assignment we tested more than one configuration so it is possible to conclude some aspects from experience and that is presented in the section
% \ref{sec:aspects}.

%To conclude is presented a final discussion about this laboratory assignment that can be seen in the section \ref{sec:conclusion}.



